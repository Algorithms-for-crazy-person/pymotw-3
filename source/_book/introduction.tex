\cleardoublepage
\chapter*{Introduction}

Distributed with every copy of Python, the Standard Library contains
hundreds of modules that provide tools for interacting with the
operating system, interpreter, and Internet – all of them tested and
ready to be used to jump-start the development of your
applications. This book presents selected examples demonstrating how
to use the most commonly used features of the modules that give Python
its ``batteries included'' slogan, taken from the popular Python
Module of the Week (PyMOTW) blog series.

\section*{This Book's Target Audience}

The audience for this book is an intermediate Python programmer, so
although all of the source code is presented with discussion, only a
few cases include line-by-line explanations.  Every section focuses on
the features of the modules, illustrated by the source code and output
from fully independent example programs.  Each feature is presented as
concisely as possible, so the reader can focus on the module or
function being demonstrated without being distracted by the supporting
code.

An experienced programmer familiar with other languages may be able to
learn Python from this book, but it is not intended to be an
introduction to the language.  Some prior experience writing Python
programs will be useful when studying the examples.

Several sections, such as the description of network programming with
sockets or hmac encryption, require domain-specific knowledge.  The
basic information needed to explain the examples is included here, but
the range of topics covered by the modules in the standard library
makes it impossible to cover every topic comprehensively in a single
volume.  The discussion of each module is followed by a list of
suggested sources for more information and further reading, including
online resources, RFC standards documents, and related books.

\section*{Python 3 versus 2}

The Python community is currently undergoing a transition from Python
version 2 to Python version 3. As the major version number change
implies, there are many incompatibilities between Python 2 and 3, and
not just in the language. Quite a few of the standard library modules
have been renamed or otherwise reorganized in the new version.

The Python development community recognized that those
incompatibilities would require an extended transition period, while
the ecosystem of Python libraries and tools was updated to work with
Python 3.  Although many projects still rely on Python 2, it is only
receiving security updates and is scheduled to be completely
deprecated by 2020. All new feature work is happening in the Python 3
releases.

It can be challenging, though not impossible, to write programs that
work with both versions. Doing so frequently requires examining the
version of Python under which a program is running and using different
module names for imports or different arguments to classes or
functions. There are tools, outside of the standard library, to make
the process simpler. To keep the examples in this book as concise as
possible, while still relying only on the standard library, they are
focused on Python 3.  All of the examples have been tested under
Python 3.5, the current release of the 3.x series at the time they
were written, and may not work with Python 2 without modification.
For examples designed to work with Python 2, refer to the Python 2
edition of the book, called \textit{The Python Standard Library By
  Example}.

Again, in an effort to maintain clear and concise descriptions for
each example, the differences between Python 2 and 3 are not
highlighted in each chapter. The Porting Notes appendix summarizes
some of the biggest differences, and is organized to be useful as an
aid when porting from Python 2 to 3.

\section*{How This Book Is Organized}

The book supplements the comprehensive reference guide available on
http://docs.python.org, providing fully functional example programs to
demonstrate the features described there. The modules are grouped into
chapters to make it easy to find an individual module for reference
and browse by subject for more leisurely exploration. In the unlikely
event that you want to read it through from cover to cover, it is
organized to minimize ``forward references'' to modules not yet
covered, although it was not possible to eliminate them entirely.

\section*{Downloading the Example Code}

The original versions of the articles and the sample code are
available at https://pymotw.com/3/.  Errata for the book can be found
on the author's web site\\
(https://www.doughellmann.com/books/byexample/).

\section*{The Conventions Used in This Book}

TBD
